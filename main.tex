\documentclass{beamer}

\mode<presentation>{\usetheme{Madrid}}
\usepackage{graphicx}
\usepackage{booktabs}
\usepackage{lmodern}
\usepackage{epstopdf}
\usepackage[ngerman]{babel}
\usepackage[T1]{fontenc}
\usepackage[latin9]{inputenc}
\usepackage{enumerate}
\usepackage{geometry}
\usepackage{tikz}

\setbeamertemplate{items}[square]

\newtheorem{proposition}[theorem]{Satz}
\newtheorem{conjecture}[theorem]{Vermutung}
\newtheorem{remark}[theorem]{Bemerkung}
\newcommand{\DF}[1]{{\bf #1\/}}

%BESONDERE SCHRIFTARTEN:

\newcommand{\cG}{{\cal G}}
\newcommand{\cE}{{\cal EG}}

%GRAPHENPARAMETER

\newcommand{\De}{\Delta}
\newcommand{\om}{\omega}
\newcommand{\al}{\alpha}
\newcommand{\cn}{\chi}

%DATUM

\def\datum{26. September 2014}

% MENGEN

\newcommand{\N}{{\mathbb{N}}}
\newcommand{\Z}{{\mathbb{Z}}}
\newcommand{\R}{{\mathbb{R}}}
\newcommand{\C}{{\mathbb{C}}}
\newcommand{\Rnn}{{\mathbb{R}^{n\times n}}}

% SONSTIGES

\newcommand{\tr}{\operatorname{trace}}
\newcommand{\set}[2]{\{#1 \;|\; #2 \}}
\newcommand{\ems}{\varnothing}
\newcommand{\sm}{\setminus}
\newcommand{\la}{\langle}
\newcommand{\ra}{\rangle}


\title[Chrom. Zahl und Spektrum von Graphen]{Chromatische Zahl und Spektrum von Graphen}
\author[Stefan Heyder]{Stefan Heyder \\ Betreuer: Prof. Dr. Stiebitz}


\institute{TU Ilmenau}
\date{30. September 2014}

\begin{document}
\begin{frame}[<+->]
  \titlepage
\end{frame}

\begin{frame}[<+->]
  \frametitle{Inhalt} 
  \tableofcontents 
\end{frame}

\begin{frame}[<+->]
  \setbeamercovered{dynamic}
  \frametitle{Ein F"arbungsproblem}
  Es sei $\cE(n)$ die Klasse aller Graphen, welche die kantendisjunkte Vereinigung von $n$ vollst"andigen Graphen der Ordnung $n$ sind.
  \pause
  \begin{problem}
    Sei $G\in \cE(n)$. Was ist die chromatische Zahl von $G$?
  \end{problem}
\end{frame}
\begin{frame}
  \setbeamercovered{dynamic}
  \frametitle{Ein F"arbungsproblem}
  \tikzstyle{vertex}=[circle,fill=black,minimum size=10pt]
\tikzstyle{edge} = [draw,line width=2pt]
\tikzstyle{green edge} = [draw,line width=2pt,darkgreen]
\tikzstyle{blue edge} = [draw,line width=2pt,blue]
\tikzstyle{red edge} = [draw,line width=2pt,red]
\tikzstyle{grey edge} = [draw,line width=2pt,yellow]

\begin{figure}
  \begin{tikzpicture}[scale=1.8, auto,swap]
    \foreach \pos/\name in { {(0,0)/a},{(0,1)/b},{(1,0)/c},{(1,1)/d}, % Erstes Quadrat
    {(1,2)/e},{(2,1)/f},{(2,2)/g}, % Zweites Quadrat
    {(2,0)/h},{(1,-1)/i},{(2,-1)/j}, % Drittes Quadrat
    {(3,0)/k},{(3,1)/l}} % Viertes Quadrat
    \node<1->[vertex] (\name) at \pos {};
    % Connect vertices with edges and draw weights
    \foreach \source/ \dest in {a/b,a/c,a/d,b/c,b/d,c/d,e/f,e/g,e/d,f/g,f/d,d/g,h/i,h/j,i/j,c/h,c/i,c/j,h/f,h/k,h/l,f/k,f/l,k/l}
    \path<1->[edge] (\source) --  node{}  (\dest);
%
%    % Start animating the vertex and edge selection. 

    \foreach \source/ \dest in {a/b,a/c,a/d,b/c,b/d,c/d}
    \path<2->[red edge] (\source) --  node{}  (\dest);
    \foreach \source/ \dest in {e/f,e/g,e/d,f/g,f/d,d/g}
    \path<2->[green edge] (\source) --  node{}  (\dest);
    \foreach \source/ \dest in {h/i,h/j,i/j,c/h,c/i,c/j}
    \path<2->[blue edge] (\source) --  node{}  (\dest);
    \foreach \source/ \dest in {h/f,h/k,h/l,f/k,f/l,k/l}
    \path<2->[grey edge] (\source) --  node{}  (\dest);

%    % For convenience we use a background layer to highlight edges
%    % This way we don't have to worry about the highlighting covering
%    % weight labels. 
%    %\begin{pgfonlayer}{background}
%    %    \pause
%    %    \foreach \source / \dest in {d/a,d/f,a/b,b/e,e/c,e/g}
%    %        \path<+->[selected edge] (\source.center) -- (\dest.center);
%    %    \foreach \source / \dest / \fr in {d/b/4,d/e/5,e/f/5,b/c/6,f/g/7}
%    %        \path<\fr->[ignored edge] (\source.center) -- (\dest.center);
%    %\end{pgfonlayer}
  \end{tikzpicture}
\end{figure}

\end{frame}
\begin{frame}
  \setbeamercovered{dynamic}
  \frametitle{Ein F"arbungsproblem}
  \tikzstyle{vertex}=[circle,fill=black,minimum size=10pt]
\tikzstyle{red vertex}=[circle,fill=red,minimum size=10pt]
\tikzstyle{blue vertex}=[circle,fill=blue,minimum size=10pt]
\tikzstyle{green vertex}=[circle,fill=green,minimum size=10pt]
\tikzstyle{grey vertex}=[circle,fill=yellow,minimum size=10pt]
\tikzstyle{edge} = [draw,line width=2pt]

\begin{figure}
\begin{tikzpicture}[scale=1.8, auto,swap]
  \foreach \pos/\name in { {(0,0)/a},{(0,1)/b},{(1,0)/c},{(1,1)/d}, % Erstes Quadrat
                     {(1,2)/e},{(2,1)/f},{(2,2)/g}, % Zweites Quadrat
                     {(2,0)/h},{(1,-1)/i},{(2,-1)/j}, % Drittes Quadrat
                     {(3,0)/k},{(3,1)/l}} % Viertes Quadrat
                     \node<1->[vertex] (\name) at \pos {};
    % Connect vertices with edges and draw weights
        \foreach \source/ \dest in {a/b,a/c,a/d,b/c,b/d,c/d,e/f,e/g,e/d,f/g,f/d,d/g,h/i,h/j,i/j,c/h,c/i,c/j,h/f,h/k,h/l,f/k,f/l,k/l}
        \path<1->[edge] (\source) --  node{}  (\dest);
%
%    % Start animating the vertex and edge selection. 

        \foreach \pos in { {(0,0)},{(1,-1)},{(1,2)},{(3,0)}}
        \node<2->[red vertex] () at \pos{};
        \foreach \pos in { {(0,1)},{(2,2)},{(2,0)}}
        \node<3->[blue vertex] () at \pos{};
        \foreach \pos in { {(1,0)},{(2,1)}}
        \node<4->[green vertex] () at \pos{};
        \foreach \pos in { {(1,1)},{(2,-1)},{(3,1)}}
        \node<5->[grey vertex] () at \pos{};

%    % For convenience we use a background layer to highlight edges
%    % This way we don't have to worry about the highlighting covering
%    % weight labels. 
%    %\begin{pgfonlayer}{background}
%    %    \pause
%    %    \foreach \source / \dest in {d/a,d/f,a/b,b/e,e/c,e/g}
%    %        \path<+->[selected edge] (\source.center) -- (\dest.center);
%    %    \foreach \source / \dest / \fr in {d/b/4,d/e/5,e/f/5,b/c/6,f/g/7}
%    %        \path<\fr->[ignored edge] (\source.center) -- (\dest.center);
%    %\end{pgfonlayer}
\end{tikzpicture}
\end{figure}

\end{frame}

\section{Die Erd\H os-Faber-Lov\'asz Vermutung}

\begin{frame}[<+->]
  \setbeamercovered{dynamic}
  \frametitle{Die Erd\H{o}s-Faber-Lov\'asz Vermutung}
  \begin{conjecture}[Erd\H{o}s-Faber-Lov\'asz(1972)]
    Sei $G\in\cE(n)$. Dann gilt $\chi(G) \leq n$.
  \end{conjecture}
  \onslide<2-> {Ein Hypergraph $H$ hei{\ss}t \DF{linear}, falls $|e\cap e'| \leq 1$ f"ur alle $e,e'\in E(H)$. }
  \onslide<3-> {
    \begin{conjecture}
      Sei $H$ ein linearer Hypergraph. Dann gilt $\chi'(H) \leq |H|$.
    \end{conjecture} 
  }
\end{frame}

\begin{frame}[<+->]
  \setbeamercovered{dynamic}
  \frametitle{Die Erd\H{o}s-Faber-Lov\'asz Vermutung}
  \begin{proposition}[Chung \& Lawler]
    F"ur jeden Graphen $G\in \cE(n)$ gilt $\chi(G) \leq \frac{3n}{2} -2$.
  \end{proposition}
  \begin{proposition}[Kahn]
    F"ur jeden linearen Hypergraphen $H$ ist $\chi'(H) \leq |H| + o(|H|)$.
  \end{proposition}
\end{frame}

\section{Krauszzerlegungen}

\begin{frame}[<+->]
  \setbeamercovered{dynamic}
  \frametitle{Krauszzerlegungen}
  Eine Menge von Untergraphen $\mathcal{K}$ von $G$ hei{\ss}t \DF{Krauszzerlegung} von $G$, falls gilt:
  \pause
  \begin{enumerate}[<+->]
    \item Alle $K\in \mathcal{K}$ sind vollst"andige Graphen der Ordnung $|K| \geq 2$.
    \item Sind $K,K'\in \mathcal{K}$ verschieden, so gilt $|K\cap K'| \leq 1$.
    \item $\bigcup\limits_{K\in \mathcal{K}} K = G$.
  \end{enumerate}
\end{frame}

\begin{frame}
  \setbeamercovered{dynamic}
  \frametitle{Krauszzerlegungen}
  \tikzstyle{vertex}=[circle,fill=black,minimum size=10pt]
\tikzstyle{edge} = [draw,line width=2pt]
\tikzstyle{green edge} = [draw,line width=2pt,green]
\tikzstyle{blue edge} = [draw,line width=2pt,blue]
\tikzstyle{red edge} = [draw,line width=2pt,red]
\tikzstyle{grey edge} = [draw,line width=2pt,yellow]

\begin{figure}
  \begin{tikzpicture}[scale=1.8, auto,swap]
    \foreach \pos/\name in { {(0,0)/a},{(0,2)/b},{(2,0)/c},{(2,2)/d}} % Erstes Quadrat
    \node<1->[vertex] (\name) at \pos {};
    % Connect vertices with edges and draw weights
    \foreach \source/ \dest in {a/b,a/c,a/d,b/c,b/d,c/d}
    \path<1->[edge] (\source) --  node{}  (\dest);
%
%    % Start animating the vertex and edge selection. 

    \foreach \source/ \dest in {a/b,a/c,b/c}
    \path<2->[red edge] (\source) --  node{}  (\dest);
    \foreach \source/ \dest in {a/d}
    \path<3->[green edge] (\source) --  node{}  (\dest);
    \foreach \source/ \dest in {b/d}
    \path<4->[blue edge] (\source) --  node{}  (\dest);
    \foreach \source/ \dest in {c/d}
    \path<5->[grey edge] (\source) --  node{}  (\dest);

%    % For convenience we use a background layer to highlight edges
%    % This way we don't have to worry about the highlighting covering
%    % weight labels. 
%    %\begin{pgfonlayer}{background}
%    %    \pause
%    %    \foreach \source / \dest in {d/a,d/f,a/b,b/e,e/c,e/g}
%    %        \path<+->[selected edge] (\source.center) -- (\dest.center);
%    %    \foreach \source / \dest / \fr in {d/b/4,d/e/5,e/f/5,b/c/6,f/g/7}
%    %        \path<\fr->[ignored edge] (\source.center) -- (\dest.center);
%    %\end{pgfonlayer}
  \end{tikzpicture}
\end{figure}

\end{frame}

\begin{frame}[<+->]
  \setbeamercovered{dynamic}
  \frametitle{Krauszzerlegungen}
  \begin{itemize}[<+->]
    \item $d_{\mathcal{K}}(v) = | \set{K\in \mathcal{K}}{v\in K}| $, der \DF{Grad} von $v$ in $\mathcal{K}$.
    \item $\delta(\mathcal{K})$, der \DF{Minimalgrad} .
    \item $\kappa_{d}(G)$, die kleinste Zahl $p$, sodass $G$ eine Krauszzerlegung $\mathcal{K}$ mit $|\mathcal{K}| = p$ und $\delta(\mathcal{K}) = d$ besitzt ($\kappa_{d}(G) = \infty$, falls kein solches $p$ existiert).
  \end{itemize}
  \visible<4> {
    \begin{figure}[h]
      \centering
      \includegraphics[width=\textwidth]{images/k5krauszdecomp}
    \end{figure} }
  \end{frame}

  \begin{frame}[<+->]
    \setbeamercovered{dynamic}
    \frametitle{Krauszzerlegungen und die Erd\H os-Faber-Lov\'asz Vermutung}
    \begin{proposition}
      Die folgenden Aussagen sind "aquivalent:
      \begin{enumerate}[<+->]
        \item F"ur alle Graphen $G\in \cE(n)$ gilt $\chi(G) \leq n$.
        \item F"ur alle linearen Hypergraphen $H$ gilt $\chi'(H) \leq |H|$.
        \item F"ur alle Graphen $G$ gilt $\chi(G) \leq \kappa_{2}(G)$.
      \end{enumerate}
    \end{proposition}
  \end{frame}

  \begin{frame}
    \setbeamercovered{dynamic}
    \frametitle{Krauszzerlegungen und die Erd\H os-Faber-Lov\'asz Vermutung}
    \definecolor{ttqqqq}{rgb}{0.2,0.0,0.0}
\begin{tikzpicture}[line cap=round,line join=round,>=triangle 45,x=1.0cm,y=1.0cm]
  \clip(1.093270212419378,-1.7484361774382082) rectangle (11.770652743098143,5.001928259024593);
  \onslide<1->{
    \draw [rotate around={-26.023263264165603:(3.2,2.83)}] (3.2,2.83) ellipse (2.14929623075802cm and 1.0200854314961239cm);
  }
  \onslide<1->{
    \draw [rotate around={62.223436191131455:(5.589999999999999,3.3100000000000005)}] (5.589999999999999,3.3100000000000005) ellipse (1.7632382490120055cm and 0.9575015001444822cm);
  }
  \onslide<1->{
    \draw [rotate around={-82.97720616442676:(5.069999999999999,0.6199999999999994)}] (5.069999999999999,0.6199999999999994) ellipse (1.678672374753102cm and 0.9405535294491401cm);
  }
  \onslide<1->{
    \draw (2.6846108780493863,3.3720874160003045) node[anchor=north west] {$e_1$};
  }
  \onslide<1->{
    \draw (5.726124892197063,3.5902550879011934) node[anchor=north west] {$e_2$};
  }
  \onslide<1->{
    \draw (4.814954027199236,1.2032440894561724) node[anchor=north west] {$e_3$};
  }
  \onslide<2->{
    \draw [color=ttqqqq] (8.446804094725787,2.7175844002976417)-- (10.320479394580474,-0.2854294364557669);
  }
  \onslide<2->{
    \draw [color=ttqqqq] (11.051982765071687,3.5004213406478897)-- (10.320479394580474,-0.2854294364557669);
  }
  \onslide<2->{
    \draw [color=ttqqqq] (11.051982765071687,3.5004213406478897)-- (8.446804094725787,2.7175844002976417);
  }
  \onslide<2->{
    \draw (8.2,2.61) node[anchor=north west] {$e_1$};
  }
  \onslide<2->{
    \draw (10.500146889087087,4.0) node[anchor=north west] {$e_2$};
  }
  \onslide<2->{
    \draw (10.41,0.19) node[anchor=north west] {$e_3$};
  }
  \onslide<2->{
    \draw (9.66597637887781,2.473749943467232) node[anchor=north west] {$K^v$};
  }
  \onslide<1->{
    \draw (4.609619747763106,-1.1195999466650577) node[anchor=north west] {$H$};
  }
  \onslide<2->{
    \draw (9.961144405567246,-1.068266376806025) node[anchor=north west] {$L(H)$};
  }
  \onslide<1->{
    \draw (4.5,2.15) node[anchor=north west] {$v$};
  }
  \onslide<2->{
    \draw [dotted] (8.446804094725787,2.7175844002976417)-- (8.278635896226707,3.18845535609506);
  }
  \onslide<2->{
    \draw [dotted] (8.446804094725787,2.7175844002976417)-- (8.141080344187905,2.3219418996015566);
  }
  \onslide<2->{
    \draw [dotted] (8.446804094725787,2.7175844002976417)-- (7.948074744810419,2.753207925291594);
  }
  \onslide<2->{
    \draw [dotted] (11.051982765071687,3.5004213406478897)-- (11.466518829484524,3.2208505065089974);
  }
  \onslide<2->{
    \draw [dotted] (11.051982765071687,3.5004213406478897)-- (11.47098180742543,3.773257926366605);
  }
  \onslide<2->{
    \draw [dotted] (11.051982765071687,3.5004213406478897)-- (10.967500857516264,3.993232468054523);
  }
  \onslide<2->{
    \draw [dotted] (10.320479394580474,-0.2854294364557669)-- (10.794821043605728,-0.443543319464195);
  }
  \onslide<2->{
    \draw [dotted] (10.320479394580474,-0.2854294364557669)-- (10.467522318999231,-0.7633189408167431);
  }
  \onslide<2->{
    \draw [dotted] (10.320479394580474,-0.2854294364557669)-- (9.96152919785495,-0.6335023544926464);
  }
  \begin{scriptsize}
    \onslide<1->{
      \draw [fill=black] (4.9,2.0) circle (1.5pt);
    }
    \onslide<2->{
      \draw [fill=ttqqqq] (8.446804094725787,2.7175844002976417) circle (1.5pt);
    }
    \onslide<2->{
      \draw [fill=ttqqqq] (10.320479394580474,-0.2854294364557669) circle (1.5pt);
    }
    \onslide<2->{
      \draw [fill=ttqqqq] (11.051982765071687,3.5004213406478897) circle (1.5pt);
    }
  \end{scriptsize}
\end{tikzpicture}

  \end{frame}
  \begin{frame}
    \setbeamercovered{dynamic}
    \frametitle{Krauszzerlegungen und die Erd\H os-Faber-Lov\'asz Vermutung}
    \definecolor{ffqqqq}{rgb}{1.0,0.0,0.0}
\definecolor{qqqqff}{rgb}{0.0,0.0,1.0}
\definecolor{qqffqq}{rgb}{0.0, 0.5, 0.0}

\begin{figure}
  \begin{tikzpicture}[line cap=round,line join=round,>=triangle 45,x=1.0cm,y=1.0cm,scale=2.1, auto,swap]
    \clip(-1.1974376215878517,-0.5015441606505369) rectangle (4.589843163638628,2.5136777610640992);

    \onslide<1->{
      \draw (-0.16399462422598035,0.051651796760817784) node[anchor=north west] {$\mathbf{G}$};
    }
    \onslide<1->{
      \draw (-1.0940933218516646,0.1) node[anchor=north west] {$\mathbf{v_1}$};
    }
    \onslide<1->{
      \draw (0.8998437554112402,0.1) node[anchor=north west] {$\mathbf{v_2}$};
    }
    \onslide<1->{
      \draw (-1.0819351689415249,2.5501522197945183) node[anchor=north west] {$\mathbf{v_4}$};
    }
    \onslide<1->{
      \draw (0.9059228318663101,2.5440731433394483) node[anchor=north west] {$\mathbf{v_3}$};
    }
    \onslide<1->{
      \draw [line width=2.4000000000000004pt] (-1.0,0.25)-- (1.0,0.25);
    }
    \onslide<1->{
      \draw [line width=2.4000000000000004pt,color=qqffqq] (1.0,2.25)-- (1.0,0.25);
    }
    \onslide<1->{
      \draw [line width=2.4000000000000004pt,color=qqqqff] (1.0,2.25)-- (-1.0,2.25);
    }
    \onslide<1->{
      \draw [line width=2.4000000000000004pt,color=qqqqff] (-1.0,2.25)-- (-1.0,0.25);
    }
    \onslide<1->{
      \draw [line width=2.4000000000000004pt,color=qqqqff] (-1.0,0.25)-- (1.0,2.25);
    }
    \onslide<1->{
      \draw [line width=2.4000000000000004pt,color=ffqqqq] (1.0,0.25)-- (-1.0,2.25);
    }
    \onslide<2->{
      \draw (3.0153623617755416,0.051651796760817784) node[anchor=north west] {$\mathbf{H}$};
    }
    \onslide<2->{
      \draw (3.67,2.28) node[anchor=north west] {$\mathbf{\textcolor{red} {K^{1}}}$};
    }
    \onslide<2->{
      \draw (2.097421817059997,1.39) node[anchor=north west] {$\mathbf{\textcolor{blue} {K^{4}}}$};
    }
    \onslide<2->{
      \draw (3.67,1.39) node[anchor=north west] {$\mathbf{\color{darkgreen} {K^{3}}}$};
    }
    \onslide<2->{
      \draw (3.67,0.5987686777171025) node[anchor=north west] {$\mathbf{K^2}$};
    }
    \onslide<3->{
      \draw (3.003204208865402,0.8176154300996165) node[anchor=north west] {$\mathbf{e_{v_1}}$};
    }
    \onslide<6->{
      \draw [rotate around={26.565051177078:(3.0121581529101342,1.7370839817477213)},line width=1pt] (3.0121581529101342,1.7370839817477213) ellipse (1.2123874734070144cm and 0.4689172482157942cm);
    }
    \onslide<4->{
      \draw (4.139991505963461,1.39) node[anchor=north west] {$\mathbf{e_{v_2}}$};
    }
    \onslide<5->{
      \draw [rotate around={0.0:(3.0121581529101404,1.2370839817477237)},line width=1pt] (3.0121581529101404,1.2370839817477237) ellipse (1.0750910294137452cm and 0.3947413349598742cm);
    }
    \onslide<5->{
      \draw (3.2,1.39) node[anchor=north west] {$\mathbf{e_{v_3}}$};
    }
    \onslide<3->{
      \draw [rotate around={-26.565051177077997:(3.0121581529101373,0.737083981747725)},line width=1pt] (3.0121581529101373,0.737083981747725) ellipse (1.2142722477903287cm and 0.47376902785406766cm);
    }
    \onslide<6->{
      \draw (3.0214414382306116,2.0516679504787922) node[anchor=north west] {$\mathbf{e_{v_4}}$};
    }
    \onslide<4->{
      \draw [rotate around={90.0:(4.01215815291014,1.2370839817477217)},line width=1pt] (4.01215815291014,1.2370839817477217) ellipse (1.1237698465665298cm and 0.51269744299359cm);
    }
    \begin{scriptsize}
      \onslide<1->{
        \draw [fill=black] (1.0,2.25) circle (3pt);
      }
      \onslide<1->{
        \draw [fill=black] (-1.0,0.25) circle (3pt);
      }
      \onslide<1->{
        \draw [fill=black] (1.0,0.25) circle (3pt);
      }
      \onslide<1->{
        \draw [fill=black] (-1.0,2.25) circle (3pt);
      }
    \end{scriptsize}
  \end{tikzpicture}
\end{figure}

  \end{frame}
  \section{Eigenwerte von Graphen}
  \begin{frame}[<+->]
    \setbeamercovered{dynamic}
    \frametitle{Eigenwerte von Graphen}
    Es sei $A(G)$ die \DF{Adjazenzmatrix} von $G$ mit 
    $$A(G)_{ij} = \begin{cases}
      1 & \text{ falls } v_iv_j \in E(G) \\
      0 & \text{ sonst.}
    \end{cases}$$
    \pause
    Die \DF{Eigenwerte} von $G$ sind dann die Eigenwerte von $A(G)$. 
    \pause
    Wir bezeichnen mit $\lambda_{i}(G)$ den \DF{$i$-gr"o{\ss}ten Eigenwert} von $G$. Also gilt 
    $$\lambda_{max}(G) = \lambda_{1}(G) \geq \lambda_{2}(G) \geq \dots \geq \lambda_{n}(G) = \lambda_{min}(G).$$
    \pause
    \begin{exmpl}
      $$\lambda_{1}(K_n) = n-1 \text{ und } \lambda_{i}(K_n) = -1, 2 \leq i \leq n$$
    \end{exmpl}
  \end{frame}


  \begin{frame}[<+->]
    \setbeamercovered{dynamic}
    \frametitle{Chromatische Zahl und Eigenwerte}
    \begin{proposition}[Wilf]
      Ist $G$ ein zusammenh"angender Graph, so gilt $\chi(G) \leq \lambda_{max}(G) +1$. Gleichheit tritt nur dann auf, wenn $G$ ein vollst"andiger Graph oder ein ungerader Kreis ist.
    \end{proposition}
    \begin{proposition}[Hoffman]
      Ist $G$ ein Graph, so gilt $\chi(G) \geq 1- \frac{\lambda_{max}(G)}{\lambda_{min}(G)}$. 
    \end{proposition}
  \end{frame}
  \begin{frame}[<+->]
    \setbeamercovered{dynamic}
    \frametitle{Krauszzerlegungen und Eigenwerte}
    \begin{proposition}
      Sei $\mathcal{K} = \{K^{1}, K^{2}, \dots , K^{p} \}$ eine Krauszzerlegung von $G$. Wir setzen $d_i = d_{\mathcal{K}}(v_i)$, wobei wir die Nummerierung der Ecken so w"ahlen, dass $d_1 \geq d_2 \geq \dots \geq d_n\geq d$ gilt.
      Dann gelten folgende Aussagen:
      \begin{enumerate}
        \item $\lambda_{i}(G) \geq -d_{n-i+1}$ f"ur alle $1 \leq i \leq n$. 
        \item $\lambda_{p+1}(G) \leq -d$, falls $p< n$.
      \end{enumerate}
    \end{proposition}
  \end{frame}

  \section{Chromatische Zahl und Eigenwerte}

  \begin{frame}[<+->]
    \setbeamercovered{dynamic}
    \frametitle{Krauszzerlegungen und Eigenwerte}

    Es seien $A= A(G)$, $D= \operatorname{diag}(d_1,d_2,\dots,d_n)$ und $B\in\R^{n\times p}$ die Inzidenzmatrix von $\mathcal{K}$. Dann gilt
    $$ B_{ij} = \begin{cases}
      1 & \text{ falls } v_i\in K^{j} \\
      0 & \text{ sonst.} 
    \end{cases} $$
    \pause
    Sei $M=BB^{T}$. Dann ist $M$ positiv semidefinit und $M=A+D$, wie sich leicht zeigen l"asst. \pause Also folgt 
    $$\lambda_{i}(A) \geq \lambda_{i}(-D) = -\lambda_{n-i+1}(D) = -d_{n-i+1}$$
    \pause
    Ist $p < n$, so ist $\operatorname{rang}(M) = \operatorname{rang} (B) \leq p < n$, insbesondere ist $\lambda_{p+1}(M) = 0$. 
    \pause
    Somit gilt
    $$\lambda_{p+1}(A) + d  \leq \lambda_{p+1}(A) + \lambda_{n}(D) \leq \lambda_{p+1}(M) = 0. $$
  \end{frame}

  \begin{frame}[<+->]
    \setbeamercovered{dynamic}
    \frametitle{Eigenwerte von Graphen}
    \begin{itemize}[<+->]
      \item F"ur $d\in \N$ sei $\xi_{d}(G) = | \set{i \in \N}{\lambda_{i}(G) > -d}|$.
      \item Es ist leicht zu zeigen, dass $\xi_{d}(G) \leq \kappa_{d}(G)$ gilt.
    \end{itemize}
    \begin{conjecture}
      F"ur alle Graphen $G$ gilt $\chi(G) \leq \xi_{2}(G)$.
    \end{conjecture}
    \begin{itemize}[<+->]
      \item Gilt diese Vermutung, so gilt auch die Erd\H os-Faber-Lov\'asz Vermutung, denn:
        \begin{itemize}[<+->]
          \item Ist $\chi(G) \leq \xi_{2}(G)$, so ist $\chi(G) \leq \kappa_{2}(G)$.
          \item Gilt $\chi(G) \leq \kappa_{2}(G)$ f"ur alle Graphen, so gilt die Erd\H os-Faber-Lov\'asz Vermutung.
        \end{itemize}
    \end{itemize}
  \end{frame}

  \begin{frame}[<+->]
    \setbeamercovered{dynamic}
    \frametitle{Graphen mit $\chi \leq \xi_{2}$}
    \pause
    Vermutung gilt f"ur 
    \pause
    \begin{itemize}
      \item $3$-f"arbbare Graphen.
      \item Kneser Graphen.
      \item Planare Graphen.
      \item Perfekte Graphen.
      \item Kantengraphen.
    \end{itemize}
  \end{frame}
  \begin{frame}[<+->]
    \setbeamercovered{dynamic}
    \frametitle{Graphen mit $\chi \leq \xi_{2}$}
    \begin{proposition}
      Sei $G$ ein Graph. Dann gilt $\chi(G) + \chi(\overline G) \leq \xi_{2}(G) + \xi_{2}(\overline G)$.
    \end{proposition}
    \begin{cor}
      Sei $G$ ein Graph. Dann gilt $\chi(G) \leq \xi_{2}(G) \text{ oder } \chi(\overline G) \leq \xi_{2}(\overline G)$.
    \end{cor}
  \end{frame}

  \begin{frame}
    \setbeamercovered{dynamic}
    \frametitle{}
    \begin{center}
      Vielen Dank f"ur Ihre Aufmerksamkeit!
    \end{center}
  \end{frame}
  \end{document}
